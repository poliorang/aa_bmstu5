\chapter*{Введение}
\addcontentsline{toc}{chapter}{Введение}

Многопоточность — способность центрального процессора (CPU) или одного ядра в многоядерном процессоре одновременно выполнять несколько процессов или потоков, соответствующим образом поддерживаемых операционной системой.

Многопоточность (как доктрину программирования) не следует путать ни с многозадачностью, ни с многопроцессорностью, несмотря на то, что операционные системы, реализующие многозадачность, как правило, реализуют и многопоточность.

Достоинства:
\begin{itemize}
	\item облегчение программы посредством использования общего адресного пространства;
	\item меньшие затраты на создание потока в сравнении с процессами;
	\item повышение производительности процесса за счёт распараллеливания процессорных вычислений;
	\item если поток часто теряет кэш, другие потоки могут продолжать использовать неиспользованные вычислительные ресурсы.
\end{itemize}

Недостатки:
\begin{itemize}
	\item несколько потоков могут вмешиваться друг в друга при совместном использовании аппаратных ресурсов \cite{Nemirovsky};
	\item с программной точки зрения аппаратная поддержка многопоточности более трудоемка для программного обеспечения \cite{Olukotun};
	\item проблема планирования потоков.
\end{itemize}


Однако несмотря на количество недостатков, перечисленных выше, многопоточная парадигма имеет большой потенциал на сегодняшний день и при должном написании кода позволяет значительно ускорить однопоточные алгоритмы.

\newpage

В лабораторной работе №3 был проведен сравнительный анализ трудоемкости трех видов сортировки: блочной, перемешиванием и бинарным деревом. Было выявлено, что блочная сортировка является независимой от входных данных: практически одинаковое время при работе с массивом чисел, отсортированным по невозрастанию, неубыванию, и массиве случайных чисел. Также сортировка зачастую обгоняла конкурентные. Но возможно ли сократить и без того небольшое время работы данной сортировки? 

Целью данной лабораторной работы является получение навыков организации параллельных вычислений на базе потоков.

Чтобы достичь намеченной цели необходимо решить следующие задачи: 

\begin{itemize}
	\item изучить понятие параллельных вычислений;
	\item изучить виды параллелизма;
	\item реализовать последовательный и параллельный алгоритм блочной сортировки;
	\item сравнить временные характеристики реализованных алгоритмов экспериментально на разных размерах входных данных;
	\item сравнить временные характеристики реализованных алгоритмов экспериментально при разном количестве рабочих потоков.
\end{itemize}

