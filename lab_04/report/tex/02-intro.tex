\chapter*{Введение}
\addcontentsline{toc}{chapter}{Введение}

Одной из задач программирования является ускорение решения вычислительных задач. 
Один из способов ее решения -- использование параллельных вычислений.

В последовательном алгоритме решения какой-либо задачи есть операции, которые может выполнять только один процесс, например, операции ввода и вывода. 
Кроме того, в алгоритме могут быть операции, которые могут выполняться параллельно разными процессами. 
Способность центрального процессора или одного ядра в многоядерном процессоре одновременно выполнять несколько процессов или потоков, соответствующим образом поддерживаемых операционной системой, называют многопоточностью \cite{multithreading}.

Процессом является программа в ходе своего выполнения. 
Каждый процесс состоит из одного или нескольких потоков -- исполняемых сущностей, которые выполняют задачи, стоящие перед исполняемым приложением. 
После окончания выполнения всех потоков процесс завершается.

Современные процессоры могут выполнять две задачи на одном ядре при помощи дополнительного виртуального ядра. 
Такие процессоры называются многоядерными. 
Каждое ядро может выполнять только один поток за единицу времени. 
Если потоки выполняются последовательно, то их выполняет только одно ядро процессора, другие ядра не задействуются. 
Если независимые вычислительные задачи будут выполняться несколькими потоками параллельно, то будет задействовано несколько ядер процессора и решение задач ускорится.

В лабораторной работе №3 был проведен сравнительный анализ трудоемкости трех видов сортировки: блочной, перемешиванием и бинарным деревом. 
Было выявлено, что блочная сортировка является независимой от входных данных: практически одинаковое время при работе с массивом чисел, отсортированным по невозрастанию, неубыванию, и массиве случайных чисел. 
Также сортировка зачастую обгоняла конкурентные. 
Но возможно ли сократить и без того небольшое время работы данной сортировки? 

Целью данной лабораторной работы является получение навыков организации параллельных вычислений на базе потоков.

Чтобы достичь намеченной цели необходимо решить следующие задачи: 

\begin{itemize}
	\item описать понятие параллельных вычислений;
	\item рассмотреть виды параллелизма;
	\item реализовать последовательный и параллельный алгоритм блочной сортировки;
	\item сравнить временные характеристики реализованных алгоритмов экспериментально на разных размерах входных данных;
	\item сравнить временные характеристики реализованных алгоритмов экспериментально при разном количестве рабочих потоков.
\end{itemize}

