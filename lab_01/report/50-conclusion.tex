\chapter*{Заключение}
\addcontentsline{toc}{chapter}{Заключение}

В результате выполнения лабораторной работы при исследовании алгоритмов нахождения расстояний Левенштейна и Дамерау-Левенштейна были применены и отработаны навыки динамического программирования, а также создания программного обеспечения для iOS.

В ходе выполнения лабораторной работы были выполнены следующие задачи: 
\begin{enumerate}[label={\arabic*)}]
	\item изучены расстояния Левенштейна и Дамерау-Левенштейна;
	\item разработаны и реализованы алгоритмы поиска расстояния Левенштейна и Дамерау-Левенштейна;
	\item создан программный продукт, позволяющий протестировать реализованные алгоритмы;
	\item проведен сравнительный анализ процессорного времени выполнения реализаций данных алгоритмов: выявлено, что реализация рекурсивного алгоритма без кэширование уступает матричному и алгоритму, использующему рекурсию с кэшем, при средней и большой длине слов, но является самым эффективным по памяти при малой длине слов;
	\item проведен сравнительный анализ затрачиваемой алгоритмами памяти: выявлено, что итеративные реализации уступают рекурсивным по данному параметру;
	\item был подготовлен отчет по лабораторной работе.
\end{enumerate}

