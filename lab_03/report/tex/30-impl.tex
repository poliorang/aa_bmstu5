\chapter{Технологическая часть}

В данном разделе производится выбор средств реализации, а также приводятся требования к программному обеспечению (ПО), листинги реализованных алгоритмов.

\section{Требования к ПО}

На вход программе подается массив сравниваемых элементов, на выходе должен быть получен отсортированный массив. Результирующий массив вычислен с помощью каждого из реализованных алгоритмов сортировки: блочной, перемешиванием, бинарным деревом. Также необходимо вывести затраченное каждым алгоритмом процессорное время и занимамую память.

\section{Выбор средств реализации}

В качестве языка программирования для реализации данной лабораторной работы был выбран язык Swift \cite{swift}. Данный язык создан компаний Apple для разработки программного обеспечения для macOS, iOS, watchOS и tvOS. Выбор обоснован желанием расширить знания в области применения данного языка, а также возможностью создать на основе написанного кода приложение.

Кроме того, в Swift есть фреймворк CoreFoundation \cite{core}, который предоставляет функции для замера процессорного времени. В качестве среды разработки выбран XCode \cite{xcode}.

\section{Листинги кода}

В листингах 3.1 -- 3.3 представлены реализации рассматриваемых алгоритмов.

\section{Листинги кода}


\hspace{0.6cm}В Листинге 3.1 показана реализация алгоритма сортировки бинарным деревом.
\begin{lstlisting}[caption=Функция алгоритма сортировки бинарным деревом]
enum BinaryTree<T: Comparable> {
    case empty
    indirect case node(BinaryTree, T, BinaryTree)
    
    func newTreeWithInsertedValue(newValue: T) -> BinaryTree {
        switch self {
        case .empty:
            return .node(.empty, newValue, .empty)
        case let .node(left, value, right):
            if newValue < value {
                return .node(left.newTreeWithInsertedValue(newValue: newValue), value, right)
            } else { return .node(left, value, right.newTreeWithInsertedValue(newValue: newValue)) }
        }
    }
    mutating func insert(newValue: T) {
        self = newTreeWithInsertedValue(newValue: newValue)
    }
    func traverseInOrder(process: (T) -> ()) {
        switch self {
        case .empty:
            return
        case let .node(left, value, right):
            left.traverseInOrder(process: process)
            process(value)
            right.traverseInOrder(process: process)
        }
    }
}

func binarySearchTreeSort<T: Comparable>(_ array: [T]) -> [T] {
    var resultArray: [T] = []
    var tree: BinaryTree<T> = .empty
    for elem in array { tree.insert(newValue: elem) }
    tree.traverseInOrder(process: { resultArray.append($0) })
    
    return resultArray
}
\end{lstlisting}

\hspace{0.6cm}В Листинге 3.3 показана реализация реализация алгоритма сортировки перемешиванием.
\begin{lstlisting}[caption=Функция алгоритма сортировки перемешиванием]
func shakerSort<T: Comparable>(_ array: [T]) -> [T] {
    var resultArray = array
    var left = 0
    var right = resultArray.count - 1
    while left <= right {
        for i in left..<right { if resultArray[i] > resultArray[i + 1] { 
        		resultArray.swapAt(i, i + 1) } }
        right -= 1
        for i in stride(from: right, to: left, by: -1) { if resultArray[i - 1] > resultArray[i] { 
        		resultArray.swapAt(i, i - 1) } }
        left += 1
    }
    return resultArray
}
\end{lstlisting}

В Листинге 3.2 показана реализация оптимизированного алгоритма блочной сортировки.
\begin{lstlisting}[caption=Функция алгоритма блочной сортировки]
func bucketSort(_ array: [Double]) -> [Double]  {
    var resultArray: [Double] = []
    let maxValue = array.max()!
    let minValue = array.min()!
    let lenArray = array.count
    let offset = array.filter { $0 < 0 }.count
    var sizeValue = maxValue / Double(lenArray) as Double
    if minValue < 0 { sizeValue = maxValue + (-minValue) / Double(lenArray) as Double }
    var buckets: [[Double]] = []
    for _ in 0..<lenArray { buckets.append([]) }
    for i in 0..<lenArray {
        let j = Int(array[i] / sizeValue)
        if j != lenArray {
            buckets[j + offset].append(array[i])
        } else { buckets[lenArray - 1].append(array[i]) }
    }
    for i in 0..<lenArray {
        insertionSort(bucket: &buckets[i])
        resultArray.append(contentsOf: buckets[i])
    }
    return resultArray
}
\end{lstlisting}

\section{Функциональные тесты}

В таблице~\ref{tbl:test} приведены тесты для функций, реализующих алгоритмы сортировки. Все тесты пройдены успешно.

\begin{table}[h!]
	\begin{center}
		\begin{tabular}{|c|c|c|}
			\hline
			Входной массив & Результат & Ожидаемый результат \\ 
			\hline
			$[15, 25, 35, 45]$ & $[15, 25, 35, 45]$  & $[15, 25, 35, 45]$\\\hline
			$[55, 45, 35, 25]$  & $[25, 35, 45, 55]$ & $[25, 35, 45, 55]$\\\hline
			$[-10, -20, -30, -25]$  & $[-30, -25, -20, -10]$  & $[-30, -25, -20, -10]$\\\hline
			$[40, -10, -30, 75]$  & $[-30, -10, 40, 75]$  & $[-30, -10, 40, 75]$\\\hline
			$[100]$  & $[100]$  & $[100]$\\\hline
			$[-20]$  & $[-20]$  & $[-20]$\\\hline
			$[1.1, 2.2, 3.3, 4.4]$  & $[1.1, 2.2, 3.3, 4.4]$  & $[1.1, 2.2, 3.3, 4.4]$\\\hline
			$[1.1, -2.2, 3.3, -4.4]$  & $[-4.4, -2.2, 1.1, 3.3]$  &  $[-4.4, -2.2, 1.1, 3.3]$\\\hline
			$[-1.1, -2.2, -3.3, -4.4]$  & $[-4.4, 3.3, -2.2, -1.1]$  &  $[-4.4, -3.3, -2.2, -1.1]$\\\hline
			$[10, 10]$  & $[10, 10]$  & $[10, 10]$ \\\hline
		\end{tabular}
		\caption{\label{tbl:test}Тестирование функций}
	\end{center}
\end{table}

\section{Вывод}

В данном разделе были разработаны исходные коды трёх алгоритмов сортировки: блочной, перемешиванием и бинарным деревом. Работа реализаций протестирована на различных наборах данных.
